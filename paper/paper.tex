% THIS IS AN EXAMPLE DOCUMENT FOR VLDB 2012
% based on ACM SIGPROC-SP.TEX VERSION 2.7
% Modified by  Gerald Weber <gerald@cs.auckland.ac.nz>
% Removed the requirement to include *bbl file in here. (AhmetSacan, Sep2012)
% Fixed the equation on page 3 to prevent line overflow. (AhmetSacan, Sep2012)

\documentclass{vldb}
\usepackage{graphicx}
\usepackage{balance}  % for  \balance command ON LAST PAGE  (only there!)
\usepackage{hyperref}
\usepackage[procnames]{listings}
\usepackage{color}

\makeatletter
\def\@copyrightspace{\relax}
\makeatother

\makeatletter
\setlength{\@fptop}{0pt}
\makeatother

\begin{document}

\definecolor{keywords}{RGB}{255,0,90}
\definecolor{comments}{RGB}{0,0,113}
\definecolor{red}{RGB}{160,0,0}
\definecolor{green}{RGB}{0,150,0}

\lstset{language=Python, 
        basicstyle=\ttfamily\small, 
        keywordstyle=\color{keywords},
        commentstyle=\color{comments},
        stringstyle=\color{red},
        showstringspaces=false,
        identifierstyle=\color{green},
        procnamekeys={def,class}}

% ****************** TITLE ****************************************

\title{Merkelized Abstract Syntax Trees}

% possible, but not really needed or used for PVLDB:
%\subtitle{[Extended Abstract]
%\titlenote{A full version of this paper is available as\textit{Author's Guide to Preparing ACM SIG Proceedings Using \LaTeX$2_\epsilon$\ and BibTeX} at \texttt{www.acm.org/eaddress.htm}}}

% ****************** AUTHORS **************************************

\numberofauthors{1}

\author{
\alignauthor Jeremy Rubin, Manali Naik, Nitya Subramanian\\
\vspace{.2cm}
       \email{\{jlrubin, mnaik, nityas\}@mit.edu} \\
       \url{https://github.com/JeremyRubin/MAST}}

\maketitle

\section{Introduction}

In the context of modern cryptosytmes, a common theme is the creation of
distributed turst networks. In most of these designs, permenant storeage of a
contract is required. However, permenant storage can become a major performance
and cost bottleneck. As a result, good code compression schemes are a key
factor in scaling these contract based cryptosystems. For this project, we
created a new data structure called the Merkelized Abstract Syntax Tree (MAST)
to address both data integrity and compression. MASTs can be used to compactly
represent contractual programs that will be executed remotely, and by using
some of the properties of Merkle trees, they can also be used to verify the
integrity of the code being executed. The project idea was developed with
Bitcoin applications in mind, and the experiment we set up uses MASTs in a
crypto currency network simulator. Using MASTs in the Bitcoin
protocol~\cite{bitcoin} would increase the complexity (length) of contracts
permitted on the network, while simultaneously maintaining the security of
broadcasted data. Additionally, contracts may contain priviledged, secret
branches of execution.


\section{MAST Data Structure}

MASTs combine the traits of Merkle Trees~\cite{merkle} and Abstract Syntax
Trees (ASTs) to compactly and securely represent programs. Merkle trees are
data structures that can be used to efficiently verify the integrity of ther
data they store.  Data blocks are stored in the leaf nodes, and every non-leaf
node is the hash of the labels of its children nodes (see Figure 1). In the
Bitcoin Blockchain, Merkle trees are currently used to efficiently store
transaction history. ASTs, on the other hand, represent the syntactic structure
of programs. Primitives are located at the leaf nodes of ASTs, and non-leaf
nodes represent programmatic operations and control flow mechanisms.
 
In a MAST, the root of the tree represents the entirety of the program, while
all other nodes represent subprograms. Each path in the tree is a different
execution branch that the program can take. The structure is “Merkelized” in
that leaf nodes are hashes of the subprogram code that they represent, and
non-leaf nodes are hashes of the children labels. MASTs can therefore compactly
represent the execution flow of a program with just a sequence of hashes that
specify which child edge to follow at each node. Overall, this means that for
a program of length $n$, a compression to $O(\log{}n)$ could be expected.

\begin{figure}[h]
	\includegraphics[scale=.4]{merkle_tree}
	\caption{An example of the construction of a Merkle Tree}
	\label{merkle}
\end{figure}

\section{Implementation}

\subsection{MAST Nodes}

A MAST node is constructed with string content and parent pointers.
The string content is code that can be executed. A MAST node can have
any number of children, each one representing a different branch of program
execution, and new branches can be added via the \texttt{addBr} method. Joining all the
string content from each node along a path in a MAST therefore yields the code
for one possible path of execution for a program.
 
Since branches in a MAST don’t have to be added in any particular order,
maintaining consistency during tree construction would be cumbersome. Instead,
that hash function of a MAST node computes the correct Merkle hash using the
current state of the tree. It does so by first forming a binary tree of all
direct children nodes. As this tree is constructed, hashes are concatenated and
hashed to compute a new Merkle hash at each level of the binary tree. Then, the
root hash of the children tree is summed with the hash of the node’s own
content, producing a Merkle hash representing the node's code and children. By
not including the node's content in the binary tree we establish a syntax-tree
style construction where the parent executes before the child. Figure 2a shows
the structure used to calculate the Merkle hash of a given MAST node. The four
children branches are placed in a binary tree whose root is the yellow ``Branch
Merkle Root.'' This Branch Merkle Root hash is summed with the content hash on
the left to give the Merkle Root.

\subsection{Proof Lists}

In order to verify the integrity of code, the MAST function
\texttt{generateFullProofUpward}  generates a proof list to a given Merkle root
hash from the current node. It does so by traversing the tree upwards,
generating a list of the Merkle hashes and code content it passes along the
way. The logic gets more complicated due to the fact that Merkle hashes are
calculated using a binary tree (as described in Section 3.1). As a result, the
proof list generator crawls up this binary tree until it hits the branch Merkle
root (the parent MAST node), and then repeats the process until the destination
node is reached. The content and hashes included in the proof list for a piece
of content is shown in Figure 2b.

For the verification process, we assume that another machine (without the
entire program code) has the Merkle hash of the MAST root node. With a proof
list whose destination node is the MAST root hash, this other machine could
verify the code in the proof list by iterating over it, summing up the hash
values to make sure that they add up to the next hash value in the proof list.
If the final summation yields the root hash, then the sequence of hashes
provided is correct. We check the integrity of the code against hashes in the
proof list as content hashes are represented.

Additionally, scripts can be compiled to a proof format that is compatible with
opcodes used on the Bitcoin Blockchain. We did not put any MAST's onto the blockchain, but tested it
via a bitcoin script interpretor we wrote. This further augments compatibility between our
MAST implementation and the Bitcoin protocol and allows for potential future
extensibility of this project to be integrated within the Bitcoin Blockchain.

\begin{figure}[h]
	\includegraphics[scale=.35]{mast}
	\caption{Structure of a simple MAST and accompanying proof. The recursive data structure (left) and a proof for a particular branch of execution for a MAST (right)}
	\label{system}
\end{figure}

\subsection{Consensus Protocol Simulation}

We implemented a crypto currency network simulator to use MASTs to verify and
execute code transferred over the network. To simulate the transaction
verification process used in Bitcoin, we created \texttt{ConsensusNode} objects to
execute and validate the code in a transaction. We implemented types
of \texttt{ConsensusNode}: \texttt{GoodNode}, \texttt{EvilNode}, and
\texttt{InconsistentNode}. \texttt{GoodNode}s execute code properly and
include transactions in their local ledgers if they are valid.
\texttt{EvilNode}s can add invalid transactions to their
ledgers and exclude valid ones. \texttt{InconsistentNode}s behave correctly with some
probability. With the three node types, we can simulate more realistic network
conditions and the presence of adversaries.\texttt{ConsensusNode} can be further
subtyped to simulate other types of adversary.


Code for a transaction is stored in a MAST, and the transaction saves the
corresponding root Merkle hash. The compression MASTs offer reduces the amount
of data that has to be transferred between nodes during validation, as well as
the amount of data that is stored in the ledgers. Rather than sending and
storing the entire transaction code, we only transmit and store the desired path through
the MAST (as a sequence of Merkle hashes). Using an args array, we support
passing arguments to transaction code and specifying the subsequent
transactions to be run. Transactions also have associated amounts that are used
to check whether the code is valid. A valid transaction is one whose amount is
at least as large as the sum of amounts of its subsequent transactions.

After individual nodes validate/invalidate a set of transactions, a special
\texttt{GlobalConsensus} node determines the final outcome of a transaction; a
transaction is valid if a majority of the \texttt{ConsensusNode}s validate it.
The \texttt{GlobalConsensus} node updates a global ledger representing the
correct state of the system, and \texttt{ConsensusNode}s sync with this
ledger at each simulation tick to maintain accurate state.

\section{Applications and Experiments}

\subsection{Contractual Agreements}


Using the previously discussed consensus protocol, we implemented a contract
modeling a will that utilized the shared contract creation, verification, and
execution functionalities of the MAST to create a multiparty execution
environment. The will-based contract we implemented consisted of an agreement
between three parties: Alice, Bob, and Carol. The branches of the tree
represent approved expenditures both before and after Alice’s death, and
present clauses of the contract as a series of branches that are unlocked upon
the fulfillments of certain conditions in conjunction with the signatures of
relevant signatories. This structure allows for sub-contracts to exist within
the subset of the primary signatories and enables full transparency of the
content of the contract while restricting execution of certain clauses to when
necessary preconditions are satisfied.

The consensus protocol to verify valid construction and execution of the will
was implemented using a group of \texttt{ConsensusNode}s (comprised of the three types
discussed above) which each verified the validity of a transaction.

\subsection{Code Compression}

The primary benchmarks used to quantify code compression were a series of
python scripts, which can be found in our repo at the location MAST/bin. The
file longcode.py (Figure 3) generates a Merkle tree with tens of thousands of
branches, yielding a total code length of 2M characters, The post-compression
result after applying our algorithms was under 23,500 characters, indicating a
compression rate of over 90%. We cannot compare the code compression we saw
directly to compressing with another algorithm such as LZW because such
algorithms employ frequency based compression and longcode uses repeated
segments again and again This is OK to do because we perform structural
compression. Instead, we compared it to using zip to compress code from the Linux kernel.
This took it from roughly 6115332 characters in length to 1754665 characters, a
compression rate of 70%. It is important to note that compression algorithms
could also be used in several places in MASTS as well: per code block, and on
the complete proof list. Additionally, we use an unoptimized message format to send
MASTS which is essentially a list of $[([subproof],data, mroot) ]$. This could be
further optimized to not use punctuation and whitespace and use more
efficient character encodings. Surprisingly, encoding the proof in a bitcoin
script (ie, self proving) was more efficient than having an external validation
script.

\begin{figure}[t!]
	\lstinputlisting[language=Python, breaklines=true]{code1.py}
	\caption{Pseudocode demonstrating code compression on a Merkle tree with millions of characters. Conversion to a MAST representation yielded a compression score of over 90\%}
           \label{code}
\end{figure}

\section{Significance and Future Work}

The primary impact of this projects is in its applications to established
environments utilizing contracts. The introduction of MASTs has potential to
greatly impact existing problems ranging from bitcoin contracts to code
transfer between distributed nodes on a network. Potential improvements to this
implementation of the MAST could include greater support for distributed
construction and execution or the addition of a framework allowing greater
extensibility by users of this data structure. Complete integration with
Bitcoin would reduce the amount of data that is stored in the Blockchain, and
will make it possible to perform more complex transactions like the will we
modeled.

\vfill

\bibliographystyle{abbrv}
\bibliography{paper}

\end{document}
